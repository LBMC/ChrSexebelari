\documentclass[10pt,a4paper]{article}
\usepackage[utf8]{inputenc}
\usepackage[english]{babel}
\usepackage{amsmath}
\usepackage{amsfonts}
\usepackage{amssymb}
\usepackage{makeidx}
\usepackage{graphicx}
\usepackage{lmodern}

\begin{document}


\section{May 17th 2018}

\subsection{Canu test with all reads of best qual that mapped to the MBELA01182 Illumina contig}
The file from \verb!Canu_tests/data!
 \verb!2018-05-17-MBELA01182_nanopore_reads_subset.fastq! was used to test canu.

The reference genome Escherichia coli OP50 was downloaded from https://www.ncbi.nlm.nih.gov/Traces/wgs/?val=ADBT01 

Command used:
 \verb!canu -p 2018-05-17-testMBELA01182 -d 2018-05-17-testMBELA01182 genomeSize=57k useGrid=false -nanopore-raw Canu_tests/data/2018-05-17-MBELA01182_nanopore_reads_subset.fastq!

Script in  \verb!src/scripts/2018-05-17-MBELA01182_canu.script!

Results in \texttt{results/2018-05-17-testMBELA01182}

\subsection{Canu test with all reads of best qual}
The file used from  \verb!Canu_tests/data!
 \verb!2018-05-17-Nanopore_all_reads_best_qual.fastq!

Command used:
 \verb!canu -p 2018-05-17-first_test_assembly -d 2018-05-17-first_test_assembly genomeSize=162.1m useGrid=false -nanopore-raw Canu_tests/data/2018-05-17-Nanopore_all_reads_best_qual.fastq!

Script in  \verb!src/scripts/2018-05-17-first_test_assembly.script!

Results in  \verb!results/2018-05-17-first_test_assembly!

ABORTED because most of the reads were too short >1000kb we will retry by changing this parameter

\subsection{Canu test with all reads of best qual test 2 changing parameters}
The file used from  \verb!Canu_tests/data!
  \verb!2018-05-17-Nanopore_all_reads_best_qual.fastq!

Command used:
 \verb!canu -p 2018-05-17-assembly_test_2 -d 2018-05-17-assembly_test_2 genomeSize=162.1m minReadLength=500 correctedErrorRate=0.16 useGrid=false -nanopore-raw Canu_tests/data/2018-05-17-Nanopore_all_reads_best_qual.fastq!

Script in  \verb!src/scripts/2018-05-17-assembly_test_2.script!

Results in  \verb!results/2018-05-17-assembly_test_2!
Job aborted

\section{May 18th 2018}

\subsection{Removing \textit{E.coli} reads from nanopore reads}

\textit{E. coli} contamination was removed from the nanopore reads:
\verb!2018-05-17-Nanopore_all_reads_best_qual.fastq!

Script in  \verb!src/data_transformations/2018-05-15-Nanopore_reads_removing_ecoli.command!

Output file \verb!results/nanopore_against_ecoli_genome/!

Nevertheless only a small part of the reads were removed, the file did not changed a lot in size after E.coli removing.

\subsection{Aligning illumina reads of female and male against g17460,g15674,g21566}

The reads of illumina cleaned from \textit{E.coli} contamination were mapped against the sequence of the 13 genes that were present in males but not in females to look for differences between females and males illumina files. Also againts RNAP2 which was used as a control for PCR experiments.

Script in  \verb!src/13genes/2018-05-15-Mapping_MRDR_illumina_against_genes.sh!

Output file \verb!results/13genes!

This was tried to do in blast but the blast db could not be created

Another try will be done with bowtie2

Script in  \verb!src/13genes/2018-05-19-Mapping_MRDR_illumina_against_genes_bowtie2.sh!

Output file \verb!results/13genes!

\section{May 19th 2018}

\subsection{Constructing Illumina reads subset}


\section{June 4st 2018}

\subsection{Hybrid assembly}

The hybrid assembly was done with the pipeline DGB2OLC. The hybrid test was performed 3 times. The parameters used for each time are present on the .sh files in results/PSMN-scripts
The first test was better than the second, where in the second the k-mers were reduced to 32 and iterations to 2. In the third test, kmers in Sparse Assembler was chaned to 63, AdapTh to 0.005 (to increase accuracy) and iterations back to 5. 

\end{document}
